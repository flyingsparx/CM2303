\documentclass[11pt,a4paper]{report}
\usepackage[margin=0.5in]{geometry}
\usepackage[explicit]{titlesec}
\usepackage[dvipsnames]{color}
\usepackage{alltt}

\definecolor{mygray}{gray}{.75}

\titleformat{name=\section,numberless}[display]
  {\normalfont\scshape\Large}
  {\hspace*{-10pt}#1}
  {-15pt}
  {\hspace*{-110pt}\rule{\dimexpr\textwidth+80pt\relax}{2pt}\Huge}
\titlespacing*{\section}{0pt}{30pt}{10pt}

\titleformat{name=\subsection,numberless}[display]
  {\normalfont\scshape}
  {\hspace*{-10pt}#1}
  {-15pt}
  {\hspace*{-110pt}\rule{\dimexpr\textwidth+30pt\relax}{0.4pt}\Huge}
\titlespacing*{\subsection}{0pt}{20pt}{5pt}


\begin{document}

\noindent\Large\textbf{CM2303 (Algorithms \& Data Structures)}\\
\noindent\large\textit{Non-assessed Labs}
\vskip30pt

\section*{Lab 1.5: Implementing basic algorithms}
\begin{enumerate}

\item Below is the outline pseudocode for the \textit{selection sort} sorting algorithm.
    \begin{alltt}
Procedure selection_sort(\(A\)):
    Input:  array of integers, \(A\)
    Output: \(A\) sorted into non-descending order
    
    \(n \gets\) length of \(A\)
    for \(i \gets 0\) to \(n - 2\) do:
        - \begin{normalfont} look through \(A\) from index \(i+1\) onwards and find the smallest item\end{normalfont}
        - \begin{normalfont} exchange the smallest item with the item at index \(i\)\end{normalfont}
    Return \(A\)
\end{alltt}
    
    \begin{enumerate}
        \item Rewrite the above outline pseudocode into detailed pseudocode (\textit{hint: change the lines of code within the for-loop}).
        \item In English, write the step-by-step instructions for \textit{selection sort} based on your detailed pseudocode.
        \item Using your detailed pseudocode, implement \textit{selection sort} in a language of your choice (ensure it can be compiled and/or run on the lab machines).
    \end{enumerate}

\item Below is the pseudocode for an algorithm for producing a `random' array of integers.
    \begin{alltt}
Procedure generate_random_array(\(l, m, n\)):
    Input:  length of array to be generated, \(l\)
            minimum integer value, \(m\)
            maximum integer value, \(n\)
    Output: random array of integers, \(A\)
    
    \(A \gets \) new array of length \(l\)
    for \(i \gets 0\) to \(l - 1\):
         \(A[i] \gets\) random integer in range \([m, n)\)
    Return \(A\)
\end{alltt}
    
    \begin{enumerate}
        \item Implement the above pseudocode as a function able to access your \texttt{selection\_sort()} algorithm from 1. (c).
        \item Test your function by writing some code to call the function and to check the resultant array's length and to ensure that each element is an integer between $m$ and $n$.
    \end{enumerate}

\item Below is the pseudocode for a basic algorithm for testing if an array has been sorted into non-descending order.
    \begin{alltt}
Procedure check_sorted_array(\(A\)):
    Input:  array of integers, \(A\)
    Output: boolean indicating sorted or non-sorted

    \(n \gets\) length of \(A\)
    for \(i \gets 0\) to \(n - 2\) do:
        if \(A[i]\) > \(A[i+1]\):
            Return \(false\)
    Return \(true\)
\end{alltt}

    \begin{enumerate}
        \item Implement the above pseudocode as a function able to access your \texttt{selection\_sort()} algorithm from 1. (c) and your \texttt{generate\_random\_array()} function from 2. (a).
        \item Test your function by writing some code to check if a random array of integers is sorted into non-descending order (\textit{hint: use your \texttt{generate\_random\_array()} function for this}). What should your function return (in most cases)?
    \end{enumerate}

\item Many major programming langauges come bundled with the ability to sort arrays of objects using some kind of sorting algorithm.
    \begin{enumerate}
        \item Using documentation or otherwise, find out how to sort an array using inbuilt functions provided by your chosen programming language.
        \item Use this function to sort a random array of length 1,000 generated by your \texttt{generate\_random\_array()} function implemented earlier.
        \item Test that the inbuilt sorting function is successful by using your \texttt{check\_sorted\_array()} function implemented earlier.
    \end{enumerate}

\item 
    \begin{enumerate}
        \item Generate a random array of integers between 1 and 10,000 of length 5,000.  
        \item Sort the random array using your \texttt{selection\_sort()} function.
        \item Test your \texttt{selection\_sort()}'s output. If your test returns \texttt{false}, you may need to fix your sorting algorithm.
    \end{enumerate}

\item Imagine you are tasked with sorting an array of students by a particular attribute.
    \begin{enumerate}
        \item Write a class (or some other kind of data structure) for a single \texttt{Student} encapsulating the following attributes:
            \begin{itemize}
                \item \texttt{student\_number} (int)
                \item \texttt{year\_group} (int)
                \item \texttt{current\_mark} (int)
            \end{itemize}
        \item Write a function that generates an instance of \texttt{Student} such that the \texttt{year\_group} is `random' in the range $[1,4)$ and the \texttt{current\_mark} is `random' in the range $[0,101)$.
        \item Write some code to use the function to generate an array of 1,000 students with unique student numbers (\textit{hint: you could use a simple iterator based on the year group for this}).
        \item Modify your \texttt{selection\_sort()} algorithm (or create a new version) to enable an array of \texttt{Student} instances to be sorted by their \texttt{current\_mark}.
        \item Modify your \texttt{check\_sorted\_array()} algorithm (or create a new version) to enable checking for students ordered by their current mark.
    \end{enumerate}

\item 
    \begin{enumerate}
        \item Write some code to use or modify your selection sort implementation to sort students from question 6. by year group and then current mark (so that students are clustered by year group, but sorted by current mark within each cluster). For example:
\begin{alltt}[
    Student: \{year_group : 1, current_mark: 60, student_number: 101\},
    Student: \{year_group : 1, current_mark: 80, student_number: 102\},
    Student: \{year_group : 2, current_mark: 30, student_number: 202\},
    Student: \{year_group : 2, current_mark: 90, student_number: 201\},
    Student: \{year_group : 3, current_mark: 70, student_number: 301\}
]
\end{alltt}
        \item Write some code to use or modify your array checker implementation to check students have been successfully sorted.
    \end{enumerate} 
        

\end{enumerate}

\end{document}
