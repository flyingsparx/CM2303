\documentclass[11pt,a4paper]{report}
\usepackage[margin=0.5in]{geometry}
\usepackage[explicit]{titlesec}
\usepackage[dvipsnames]{color}
\usepackage{graphicx}
\usepackage{alltt}

\definecolor{mygray}{gray}{.75}

\titleformat{name=\section,numberless}[display]
  {\normalfont\scshape\Large}
  {\hspace*{-10pt}#1}
  {-15pt}
  {\hspace*{-110pt}\rule{\dimexpr\textwidth+80pt\relax}{2pt}\Huge}
\titlespacing*{\section}{0pt}{30pt}{10pt}

\titleformat{name=\subsection,numberless}[display]
  {\normalfont\scshape}
  {\hspace*{-10pt}#1}
  {-15pt}
  {\hspace*{-110pt}\rule{\dimexpr\textwidth+30pt\relax}{0.4pt}\Huge}
\titlespacing*{\subsection}{0pt}{20pt}{5pt}


\begin{document}

\noindent\Large\textbf{CM2303 (Algorithms \& Data Structures)}\\
\noindent\large\textit{Non-assessed Labs}
\vskip30pt

\section*{Lab 2.4: Branch and Bound: The Knapsack Problem}

The knapsack problem is a famous optimisation problem, in which there are a set of items to be placed in a knapsack of specified weight capacity. Each item has a weight and a profit, and the problem is focussed around filling the knapsack as much as possible whilst maximising profit.

In this session we will use three different branch-and-bound algorithms for solving this problem. Please refer to your lecture notes for more information on branch-and-bound algorithms.

Lab 2.4 has no prerequisite labs.

\begin{enumerate}

    \item Write a class called \texttt{Item} to represent an item to place in the knapsack. \texttt{Item} should have a constructor that accepts two integer arguments: a weight and a profit, and public methods for getting the item's weight and profit.

    \item Write a public method for \texttt{Item} that returns the profit-weight ratio of the item. Consider which return type you might need for this method.

    \item Write a class called \texttt{Knapsack} to represent the knapsack in which items are placed. This class should maintain a list of \texttt{Item}s and have a constructor that initialises the knapsack with an integer maximum capacity.

    \item Write two public methods for \texttt{Knapsack} - \texttt{get\_weight()} and \texttt{get\_profit()} - that respectively return the total weight and profit of the knapsack.

    \item Write a public method for \texttt{Knapsack} that accepts a single \texttt{Item} object and adds it to the knapsack. You might like to place a check inside this method to see if the knapsack weight has exceeded capacity.

    \item Below is the outline pseudocode for finding the lower bound of the branch and bound algorithm for a given capacity and a list of items to add to the knapsack. This method sorts the input list of items by profit-weight ratio in order to find those that are the most valuable. It then adds one at a time to the knapsack for as long as its total weight is less than its capacity. The item that would `break' the knapsack is known as the \textit{break item}.
\begin{alltt}
procedure branch\_and\_bound\_lower:
    Inputs: \(capacity\) (int), \(items\) (list)
    Output: filled knapsack, \(k\)

    \(sorted \gets items\) sorted in order of profit-weight ratio
    \(k \gets\) knapsack initialised with capacity \(capacity\)
    
    for \(item \in sorted\):
        if \(k\)'s weight + \(item\)'s weight \(\leq capacity\):
            add \(item\) to \(k\)

    return \(k\)
\end{alltt}
    Implement this algorithm by writing a method called \texttt{branch\_and\_bound\_lower()}. This method will need access to your \texttt{Knapsack} and \texttt{Item} classes.

    \item The method \texttt{branch\_and\_bound\_lower()} could be improved: if there are item(s) beyond the break item in the $sorted$ list that would fit in the knapsack, then they may as well be added. They might not have the highest profit-weight ratio of those left, but at least the total profit of the knapsack items would be increased.

    Write a new method, called \texttt{branch\_and\_bound\_lower\_improved()}, that implements this improvement.

    \item Below is a table of items, each with a weight and a profit value.\\
    \begin{center}
    \begin{tabular}{ c | c }
        Weight & Profit\\
        \hline
        \hline
        42 & 92\\
        45 & 57\\
        29 & 49\\
        50 & 50\\
        53 & 60\\
        80 & 40\\
        38 & 43\\
        63 & 67\\
        85 & 84\\
        89 & 87\\
        82 & 72\\
        \hline
    \end{tabular}
    \end{center}
    In a main method, or elsewhere, create a list of \texttt{Item}s corresponding to the above table.

    \item Using a capacity of 165, pass your list of \texttt{Item}s to your two branch-and-bound methods in order to fill the knapsack. Then verify your resultant knapsack objects by comparing their weights and profits to the values below.

    \textbf{Lower bound:}\\
    Total weight: 116\\
    Total profit: 198

    \textbf{Lower bound improved:}\\
    Total weight: 154\\
    Total profit: 241\\

    Notice that the weight and profit of the knapsack is greater in the improved approach.
    
    \item In our improved lower-bound method we take an item (or multiple items in some cases) beyond the break item that is/are able to fit in the knapsack. However, the item(s) added do not have the best profit-weight ratio. An upper bound for the branch-and-bound algorithm can be calculated by instead only adding a \textit{proportion} of the break item such that it completely fills the knapsack. This means that we only focus on the items that have the highest profit-weight ratio, and thus our knapsack will convey maximum profit.

    In cases where the resultant proportionate profit is not an integer, we \textit{round down} to the nearest int.

    Write a new method, called \texttt{branch\_and\_bound\_upper()}, that implements this change.
    
    \item Verify that your upper-bound method works by passing the same list of \texttt{Item}s and capacity to the method and comparing the resultant knapsack to the below values.

    \textbf{Upper bound:}\\
    Total weight: 165\\
    Total profit: 253

    Notice that the knapsack's total weight is now equal to its capacity. The profit is now maximised for this instance of the knapsack problem.

\end{enumerate}

\end{document}
