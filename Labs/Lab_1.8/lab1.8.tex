\documentclass[11pt,a4paper]{report}
\usepackage[margin=0.5in]{geometry}
\usepackage[explicit]{titlesec}
\usepackage[dvipsnames]{color}
\usepackage{enumitem}
\usepackage{alltt}

\definecolor{mygray}{gray}{.75}

\titleformat{name=\section,numberless}[display]
  {\normalfont\scshape\Large}
  {\hspace*{-10pt}#1}
  {-15pt}
  {\hspace*{-110pt}\rule{\dimexpr\textwidth+80pt\relax}{2pt}\Huge}
\titlespacing*{\section}{0pt}{30pt}{10pt}

\titleformat{name=\subsection,numberless}[display]
  {\normalfont\scshape}
  {\hspace*{-10pt}#1}
  {-15pt}
  {\hspace*{-110pt}\rule{\dimexpr\textwidth+30pt\relax}{0.4pt}\Huge}
\titlespacing*{\subsection}{0pt}{20pt}{5pt}


\begin{document}

\noindent\Large\textbf{CM2303 (Algorithms \& Data Structures)}\\
\noindent\large\textit{Non-assessed Labs}
\vskip30pt

\section*{Lab 1.8: Sets and bags}

It is worth reading up on sets and multisets before starting this worksheet in order to more fully understand their function and features.

\vskip20pt

\noindent \textbf{Definition}\\
A \textit{set} is an abstract data type that stores an arbitrary number of unordered unique values.

\vskip20pt

\begin{enumerate}

    \item In Java, write the skeleton of a class called \texttt{MySet} that will be your own representation of a set of integers.

    \item Declare a private array in \texttt{MySet} called \texttt{set} to represent your `set of integers'.

    \item Write a method for \texttt{MySet} called \texttt{add()} that accepts a single integer and adds it to the set.

    \item Write a method for \texttt{MySet} called \texttt{remove()} that accepts a single integer argument and removes a matching integer from the set.

    \item Write a method for \texttt{MySet} called \texttt{get()} to return the current value of \texttt{set}.

    \item Write a constructor for \texttt{MySet} that accepts an array of integers as an argument and uses \texttt{add()} to populate \texttt{set}. Add a second constructor that accepts no argument, but still instantiates \texttt{set}.

    \item Write a method for \texttt{MySet} called \texttt{print()} that prints a comma-separated \texttt{set} to standard output.

    \item Test \texttt{MySet} by writing some code to check its construction.
\begin{alltt}
s = MySet(\{4, 2, 1, 3, 5, 2\});
s.print(); // Should print "4, 2, 1, 3, 5" - order is unimportant
s.add(1);
s.add(6);
s.print(); // Should print "4, 2, 1, 3, 5, 6" - order is unimportant
\end{alltt}
    
    \item Write a method called \texttt{union()} that accepts a single \texttt{MySet} argument and returns the union between the present instance  and the argument.

    \item Write a method called \texttt{intersection()} that accepts a single \texttt{MySet} argument and returns the intersection between the present instance and the argument.

    \item Write a method called \texttt{difference()} that accepts a single \texttt{MySet} argument and returns the difference between the present instance and the argument.

    \item Write a method called \texttt{subset()} that accepts a single \texttt{MySet} argumant and returns \textit{true} if the argument is a subset of the present instance, and \textit{false} if otherwise.

    \item Write some code to test the more advanced set functionality implemented in questions 9-12.
\end{enumerate}

\noindent \textbf{Definition}\\
\noindent A \textit{multiset} (or \textit{bag}) is an abstract data type, similar to a set, but which allows for repeated equal values by maintaining a count of any duplicates it contains.

\vskip20pt

\begin{enumerate}[start=14]
    
    \item Write a class called \texttt{Student} which contains two public fields - name (string) and year group (integer).
    
    \item Write a class called \texttt{StudentSet} to represent a multiset of \texttt{Student}s. Write the \texttt{add()} and \texttt{remove()} methods for \texttt{StudentSet}, which should maintain a counter for the number of `equal' students it holds. You may like to use a dictionary-type structure for this. (Don't worry about implementing any further set functionality for this class.)

    \item Write a method for \texttt{StudentSet} called \texttt{count()} that accepts a string (name) and integer (year group). The method should return an integer representing the number of `matching' \texttt{Student}s in the multiset.

    \item Write a method for \texttt{StudentSet} called \texttt{contains()} that accepts a \texttt{Student} object and returns \textit{true} if the multiset contains a `matching' \texttt{Student} and \textit{false} if otherwise.

    \item Write a method for \texttt{StudentSet} called \texttt{scaled\_by()} that accepts a float argument. Calling this method should not change the \texttt{Student}s represented by the multiset, but should scale the \textit{number} of each by the argument amount.

    \item Test your multiset by writing some code to instantiate a few \texttt{Student} objects and inserting them into the multiset.

\end{enumerate}
\end{document}
