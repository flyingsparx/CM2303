\documentclass[11pt,a4paper]{report}
\usepackage[margin=0.5in]{geometry}
\usepackage[explicit]{titlesec}
\usepackage[dvipsnames]{color}
\usepackage{enumitem}
\usepackage{alltt}

\definecolor{mygray}{gray}{.75}

\titleformat{name=\section,numberless}[display]
  {\normalfont\scshape\Large}
  {\hspace*{-10pt}#1}
  {-15pt}
  {\hspace*{-110pt}\rule{\dimexpr\textwidth+80pt\relax}{2pt}\Huge}
\titlespacing*{\section}{0pt}{30pt}{10pt}

\titleformat{name=\subsection,numberless}[display]
  {\normalfont\scshape}
  {\hspace*{-10pt}#1}
  {-15pt}
  {\hspace*{-110pt}\rule{\dimexpr\textwidth+30pt\relax}{0.4pt}\Huge}
\titlespacing*{\subsection}{0pt}{20pt}{5pt}


\begin{document}

\noindent\Large\textbf{CM2303 (Algorithms \& Data Structures)}\\
\noindent\large\textit{Non-assessed Labs}
\vskip30pt

\section*{Lab 1.9: More algorithmic complexity}

As we saw in Lab 1.7, complexity can be measured by timing how long it takes an algorithm to complete a task. Complexity can also be measured by counting the number of \textit{key operations} the algorithm performs. If you have not finished Lab 1.7, please do so first.

\vskip20pt

\begin{enumerate}

    \item Modify your \texttt{largest\_element()} method from Lab 1.7 so that it sets an integer to 0 and then increments it each time \texttt{max} is reassigned. Your method should be either able to return this count variable, or modify an existing variable in the method's scope, so that the integer is available outside of this method.

    \item Using your \texttt{generate\_random\_array()} method from Lab 1.5, or otherwise, generate a random array of 10,000 integers long.

    \item Using a random array of length 10,000, compute the number of operations required in the execution of \texttt{largest\_element()}. Is the count the same each time a new random array is generated?

    \item Write a loop to repeat the computation a few times and print the mean number of operations to standard output. Choose a number of repeats such that the mean is relatively consistent each time. (\textit{Hint: if you are using a scoped integer variable to count, then ensure this is reset to 0 on each repeat})

    \item Write an extra outer loop to gradually increase the length of the array, in intervals of 5,000, up until a length of 100,000 is reached. In each iteration, the mean operations count for each length should be output.

    \item Using a GUI graphing package, or otherwise, plot the mean operations count against the array length. Compare the shape of the plot to that produced for this algorithm in Lab 1.7. 

    \item Repeat questions 3-6 for your \texttt{selection\_sort()} algorithm from Lab 1.5.

    \item What are the advantages of this method of measuring complexity over that used in Lab 1.7?

    \item Let \texttt{f(n)} be a call to a method that performs precisely \textit{n} operations. Express the complexity of the algorithms below in big-O notation.
    \begin{enumerate}
        \item
\begin{alltt}
for(int i = 0; i < n; i++)\{
    f(n);
\}
\end{alltt}

        \vskip20pt
        \item 

\begin{alltt}
f(n);
f(n);
for(int i = 0; i < n; i++)\{
    for(int j = 0; j < n; j++)\{
        f(n\(\sp{2}\))
    \}
\}
\end{alltt}

    \vskip20pt
    \newpage
    \item

\begin{alltt}
for(int i = 0; i < n; i++)\{
    for(int j = 0; j < m; j++)\{
        for(int k = 0; k < n; k++)\{
            f(p);
        \}
    \}
\}
\end{alltt}

    \end{enumerate}
\end{enumerate}
\end{document}
