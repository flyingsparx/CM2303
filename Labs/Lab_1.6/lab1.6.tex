\documentclass[11pt,a4paper]{report}
\usepackage[margin=0.5in]{geometry}
\usepackage[explicit]{titlesec}
\usepackage[dvipsnames]{color}
\usepackage{alltt}

\definecolor{mygray}{gray}{.75}

\titleformat{name=\section,numberless}[display]
  {\normalfont\scshape\Large}
  {\hspace*{-10pt}#1}
  {-15pt}
  {\hspace*{-110pt}\rule{\dimexpr\textwidth+80pt\relax}{2pt}\Huge}
\titlespacing*{\section}{0pt}{30pt}{10pt}

\titleformat{name=\subsection,numberless}[display]
  {\normalfont\scshape}
  {\hspace*{-10pt}#1}
  {-15pt}
  {\hspace*{-110pt}\rule{\dimexpr\textwidth+30pt\relax}{0.4pt}\Huge}
\titlespacing*{\subsection}{0pt}{20pt}{5pt}


\begin{document}

\noindent\Large\textbf{CM2303 (Algorithms \& Data Structures)}\\
\noindent\large\textit{Non-assessed Labs}
\vskip30pt

\section*{Lab 1.6: Implementing a basic data structure: String}
\begin{enumerate}

    \item Choose a programming language runnable/compilable on the lab machines and create a new class (or data structure) named \texttt{MyString}. This class will become your own representation of the String data type, common to many languages, using an array of single characters. In your class, declare a private (if available) character array to represent your string. Depending on your language, your class may look something like this:
\begin{alltt}
class MyString \{
    \textbf{private} char \textit{array} myString;
\}
\end{alltt}

    \item Write a constructor method for your \texttt{MyString} class that will enable you to instantiate new objects of \texttt{MyString}. Your constructor should accept one single standard/native String argument: an initial starting value for your \texttt{MyString} object.
    
    \item Modify your constructor method so that it instantiates your \texttt{myString} array to the correct length and stores the constructor's String argument as a series of characters represented by \texttt{myString}.

    \item Overload or modify your constructor (depending on language) so that objects of your class can be instantiated \textit{without} an argument. The \texttt{myString} array will still need to be initialized.

    \item Create a (private) method for your class, called \texttt{get\_string()}, that returns the \texttt{myString} character array as a standard/native String.

    \item Create a (public) method for your class, called \texttt{print()}, that appropriately prints the represented String to standard output. (\textit{hint: your \texttt{get\_string()} method might be useful}).

    \item Create a (public) method for your class, called \texttt{equals()}, that accepts a single String argument and returns \textit{true} if the argument String has an equal value to your String representation, and \textit{false} if otherwise.

    \item Overload or modify \texttt{equals()} so that it can also accept instances of \texttt{MyString} to check for equality.

    \item Modify your \texttt{equals()} method to accept a second Boolean argument that specifies whether, or not, case should be ignored when checking for equality.

    \item Check your \texttt{MyString} class so far by writing some simple tests:
\begin{alltt}
s = new MyString("Hello, world");
case_insensitive_match = s.equals("hello, world", false); // Should be true
case_sensitive_match = s.equals("hello, world", true); // Should be false
\end{alltt}

    \item Create a (public) method for your class, called \texttt{get\_length()}, that returns the current length of your String representation as an Integer.

    \item Write a method, \texttt{index\_of()}, that accepts a single character \textbf{or} character sequence. The method should return the Integer index of the first occurrence of the argument. If no match is found, the method should return -1. (\textit{hint: You may need to overload your method}).

    \item Write a method, \texttt{contains()}, that accepts a single character \textbf{or} character sequence. The method should use \texttt{index\_of()} to return \textit{true} if the character or sequence exists within the String representation, or \textit{false} if otherwise.

    \item Write a method, \texttt{get\_character()}, that accepts a single Integer, $i$, and returns the character of the String at position $i$. 
 
    \item Write a method, \texttt{insert()}, that accepts an Integer index and a character and inserts the specified character into that position of \texttt{myString}. (\textit{hint: you will need to remember to extend \texttt{myString} and shift later characters along by one}).

    \item Write a method, \texttt{delete()}, that accepts a single Integer index and deletes the character at that position of the array. 

    \item Overload (or modify) \texttt{delete()} so that it also accepts a character. The method should use your \texttt{index\_of()} method to appropriarely delete the first instance of the specified character.

    \item Overload (or modify) your \texttt{insert()} and \texttt{delete()} method so that they can both also accept character sequences for insertion and deletion.

    \item Overload (or modify) your \texttt{delete()} method so that it also accepts a second Integer argument to indicate the number of characters to be deleted from the specified location. For example, calling \texttt{delete(4, 3)} on the \texttt{MyString} instance representing \texttt{Hello, world} should result in the instance instead representing \texttt{Hellworld}.

    \item Using \texttt{insert()}, overload or modify your constructor so that new instances of \texttt{MyString} can be initialized from existing \texttt{MyString} objects.

    \item Your \texttt{MyString} class now has some rich String functionality similar to that often provided by major programming languages. Test your class by writing some code to instantiate it and invoke its methods.

\begin{alltt}
MyString s = new MyString();
s.print(); // Should print an empty string

s = new MyString("Cardiff");
int l = s.get_length(); // Should be equal to 7
int i = s.index_of("d"); // Should be equal to 3
Boolean c = s.contains("Car"); // Should be true
Boolean c2 = s.contains("car"); // Should be false
s.insert("city of ", 0);
s.print(); // Should print 'city of Cardiff'

MyString s2 = new MyString(s);
Boolean e = s.equal(s2); // Should be true
int l2 = s2.get_length(); // Should be 15
\end{alltt}
\end{enumerate}
\end{document}
